\chapter{Conclusion}

While previous discussion of the Force Network Ensemble had been abstractly defined as a particular region in force space, our work in Chapter \ref{forceVolumeEntropyPaper} was the first to measure this space in a concrete way. Moreover, we were able to connect our measurements of the Force Network Ensemble with a bulk measurement of angoricity. In doing so we not only discovered a new quantity, $h_f$ which sets the scale of discretization for these systems, but also put the concept of angoricity on firm footing as a thermodynamic quantity. This connects thermodynamic approaches to granular materials with an underlying statistical mechanics, contributing to a complete statistical mechanical description of granular materials.

In Chapter \ref{contactBreaking} we expanded on this work, using the Force Network Ensemble to identify systems at the boundaries of available force space. From this we devised a provably correct first principles method for analytically identifying the only locations at which any contact change can occur under decompression. These force network defects are a completely new form of defect in amorphous materials and provide a framework for a ground up understanding of amorphous materials.

Despite the quantity of questions which have been answered by this work, an even greater number of interesting new questions have been raised. We hope that our approach of studying the space of allowed force configurations in granular systems serves as the foundation for future research. In particular, it would be interesting to use our strategy from Chapter \ref{contactBreaking} to predict contact breaking events in systems further from jamming via the Force Network Ensemble. While we limited our study to 2SSS systems, the same principle should apply to systems with more states of self stress. The problem becomes more challenging as the dimensionality of the space of states of self stress increases, but the same protocol can be used on a system with many states of self stress to predict not just the breakable contacts, but also all of the allowed 1SSS final configurations after decompression. Of course, since some contact breaking events are rearrangements, predictions for sequential contact breaking events will be less accurate. However, since about $86\%$ of contact network changes are stable contact network changes, we would still expect to have decent predictive power for at least a few contact breaks \cite{morse_differences_2020,tuckman_contact_2020}.

Another interesting and unexplored avenue is the search for force network defects in ultrastable systems \cite{kapteijns_fast_2019,hagh_transient_2021,ninarello_models_2017}, which have more contacts near jamming than random packings, and as such the defects may be delocalized or poorly defined. One could also attempt to modify our protocol to identify force network defects under shear rather than decompression. This would be especially interesting because the majority of the work done on soft spots has been conducted under shear, as were the recent studies which distinguished and quantified rearrangements versus stable contact network changes \cite{morse_differences_2020,tuckman_contact_2020}.

Finally, one can also envision attempting to ascertain which force network defects will be rearrangements versus stable contact network changes. This could concievably be done with machine learning, in a similar fashion to much of the soft spots literature. Perhaps only the force network defects which exist within identifiable soft spots tend to lead to rearrangements, and others tend to lead to stable contact network changes. 

In Chapter \ref{excessContactsScaling} we took highly precise measurements of jamming scaling laws to measure the prefactors to jamming scaling relations. We showed that the mean field prediction is accurate for not just scaling exponents, but even the prefactors to these scaling laws. This was quite surprising because we expect these prefactors to be nonuniversal and highly sensitive to finite dimensional corrections to mean field theory. We were additionally able to show a first principles proof for the scaling prefactor between pressure and packing fraction. This showed that the mean field assumption was not necessary for that particular relation, and suggests that other relations may similarly be proven without the need for mean field theory. This calls into question the necessity of the mean field assumption in general for glasses and jamming. In particular, is there a derivation for the prefactor between excess contacts and pressure that does not rely on the mean field assumption? Moreover, are other scaling prefactors well predicted by the mean field?  Of course, we do not expect that all useful results from the mean field may be proven without the mean field assumption - but perhaps more can be than previously thought. 

The work presented in this dissertation represents a substantial addition to the literature regarding the Force Network Ensemble for jammed amorphous systems. We have shown two new methods by which the Force Network Ensemble can be used to better understand granular systems. Additionally we have shown that jamming scaling laws can be proven from first principles without the mean field assumption, suggesting that the mean field assumption may not be necessary for understanding scaling laws in jamming.
