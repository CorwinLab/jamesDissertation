\chapter{Conclusion}

The work presented in this dissertation represents a substantial addition to the literature regarding the Force Network Ensemble for jammed amorphous systems. While previous discussion of the Force Network Ensemble had been abstractly defined as a particular region in force space, our work in Chapter \ref{forceVolumeEntropyPaper} was the first to measure this space in a concrete way. Moreover, we were able to connect our measurements of the Force Network Ensemble with a bulk measurement of angoricity. In doing so we not only discovered a new quantity, $h_f$ which sets the scale of discretization for these systems, but also put the concept of angoricity on firm footing as a thermodynamic quantity. This connects thermodynamic approaches to granular materials with an underlying statistical mechanics, contributing to a complete statistical mechanical description of granular materials.

In Chapter \ref{contactBreaking} we expanded on this work, using the Force Network Ensemble to identify systems at the boundaries of available force space. From this we devised a provably correct first principles method for analytically identifying the only locations at which any contact change can occur under decompression. These force network defects are a completely new form of defect in amorphous materials and provide a framework for a ground up understanding of amorphous materials.

In Chapter \ref{excessContactsScaling} we took highly precise measurements of jamming scaling laws to measure the prefactors to jamming scaling relations. We showed that the mean field prediction is accurate for not just scaling exponents, but even the prefactors to these scaling laws. This was quite surprising because we expect these prefactors to be highly sensitive to finite dimensional corrections to mean field theory. We were additionalyl able to show a first principles proof for one of these scaling prefactors. This showed that the mean field assumption was not necessary for that particular relation, and suggests that other relations may similarly be proven without the need for mean field theory.
