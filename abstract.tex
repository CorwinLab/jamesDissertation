\abstract{
abstract here
%A vast variety of physical systems falls within the description of amorphous solids. From glasses to grains, all of these materials share a disordered structure of their constituents. Understanding the nature of the mechanical properties of such systems is a conundrum which still poses challenging open questions. Recent experimental advances have led to the conclusion that the preparation of the system controls its stability against mechanical perturbations. In particular, amorphous solids can be classified as marginally stable or highly stable with respect to external perturbations. In this work I show that the amorphous structure, whether marginally or highly stable, uniquely controls the mechanical response of amorphous solids. First, I show that thermal glasses under very high pressure share the same mechanical and vibrational properties of athermal granular packings near the onset of rigidity. Secondly, I investigate the role of mechanical stability in the context of rheology, in particular with respect to cyclic shear training, and show that jammed solids are able to store an information of the repeated shear deformation only if the system, or a portion of it, is marginally stable.

This dissertation includes previously published and unpublished coauthored material.}
