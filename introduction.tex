\chapter{Introduction}

When you think of solids, you generally think of crystals. You take a liquid, and let it cool down, and it falls into a nice crystal structure that is spatially optimized in some way. This is generally how most solids form and is conceptually pretty easy to understand. Glass however can be formed by cooling down a liquid too quickly to find its minimum energy configuration - it essentially just gets stuck in some non-optimal state instead. Granular materials exist in a similar way - if you've ever stacked oranges, you can arrange them quite efficiently in a FCC lattice. But if you pour them together, they will be arranged randomly. If you then squeeze the box, the oranges don't spontaneously form the FCC lattice - they are frustrated in the sense that they are all blocking each other from forming this more optimal arrangement, in the sense that you could fit more oranges in if they were arranged in a lattice. In contrast to crystalline solids, amorphous solids and granular materials are generally extremely difficult to understand from first principles. The only successful approach is in the limit of infinite spatial dimensions, termed ``mean field.'' This seems kind of absurd, and it is - results from the mean field limit are not certain to aplpy to any finite dimensional system. Many results from the mean field however are shockingly accurate even in as low as 2 and 3 dimensions.

A phase transition is when a material undergoes a discontinuous change in some property (the ``order parameter'') as some other property (the ``control parameter'') varies. The most well known phase changes are freezing/melting and boiling/condensating, for which the temperature is the control parameter, and the density of the material can be seen to discontinuously change at critical temperatures. The order parameter is generally accepted to be the free energy - at the phase transition, melting absorbs a latent heat separate from the heat required just to heat it to the melting point. While glasses go through thermal phase transitions, granular materials do not. Granular materials are thus described as being ``athermal'' or ``zero-temperature.'' If you imagine the box of oranges as described earlier, thermal fluctuations (with energy ~kT) will never reach the temperature required to rearrange a couple of oranges (energy ~ mgr). Granular materials do however go through a different type of transition, called the jamming and unjamming transitions, which are similar to typical phase transitions in some ways, but different in other ways. The control parameter for the jamming transition is the density $\varphi$, and the jamming transition happens at a particular density where properties of the system go through discontinuous changes. In particular, the pressure $P$ goes from 0 to a finite value, and the number of force bearing contacts between granules goes from 0 to roughly the spatial dimension times the number of particles $Nd$. The jamming density $\varphi_j$ is an exact value for any particular system, but generally has some variation between systems, and can vary based on the protocol of system generation. The jamming point can be understood as a critical point, and as the system increases in density from that point, properties such as the number of contacts in excess of ND, the pressure, and the density in excess of $\varphi_j$ all scale with each other as power laws rather than having an absolute scale (e.g. exponential scalings).

Traditional approaches to analyzing thermodynamic systems rely on the assumption of ergodicity, i.e. that the thermal system explores the ensemble of all available configurations. The probability of existing in any particualar configuration is related to the energy of that configuration compared with the temperature of the system. In a zero temperature granular material, however, this is obviously impossible. Despite this, granular materials respond predictably to repeated disturbances, e.g. if you repeatedly shake a box of rocks, they will reliably settle into a configuration of roughly the same density. Thus some version of athermal statistical mechanics is clearly at work. In the 80's Sir Sam Edwards begun the first serious attempt at understanding granular materials in thermodynamic terms. The parameters compactivity and angoricity came about as analogs to temperature that relate to the density and pressure rather than temperature. While this approach has had some success, measurement of angoricity and compactivity is difficult.

The jamming transition describes the onset of rigidity. Below $\varphi_j$, a system is floppy and unable to sustain any force. Above $\varphi_j$ however a system becomes rigid and can support external forces. You can run your hands through sand at the beach, but if you compress it, it will push back. This rigidity can be understood in terms of degrees of freedom and constaints - it arises when the system has at least as many constraints as degrees of freedom. A system where these are exactly equal is termed ``isostatic,'' and a system with more constraints than degrees of freedom is termed ``hyperstatic.'' Since each particle can move in $d$ dimensions, a granular system has $Nd$ degrees of freedom and the constraints to those degrees of freedom are the contacts between particles. All of our jammed configurations that we study are hyperstatic and thus are rigid and can support external forces. When considering the force network, each contact is a degree of freedom, and requiring force balance on each particle imposes $Nd$ constraints. Thus while the positions of particles in a hyperstatic system is overdetermined, the force network supporting this rigidity is underdetermined. There exists a degerate space of allowed force configurations, which is referred to as the ``Force Network Ensemble.'' This was first formulated by some dutch guys in 2004, and we thought it was cool so we used it a lot. 

The following three chapters are the three papers which we composed during my time as a graduate student. The first explores a method that we developed that uses the Force Network Ensemble to measure the allowed space of force configurations. This allowed us to measure the angoricity and thus the entropy of the force networks of a granular system. In the second, we use a similar approach to examine the boundaries of the allowed space of force configurations. These boundaries correspond to systems with fewer contacts than the original system. By examining which contacts are missing in those systems, we are able to predict which contacts between particles are unnecessary, and show that under decompression of a system, these contacts are in fact the ones that are lost. In the third, we closely examine the relation between the critical power laws from the jamming transition. Rather than looking at the exponents of these power laws, which are quite well known, we explore instead the prefactors to these power laws. Although prefactors are highly sensitive to finite dimensional corrections to mean field, we find that they are still well predicted by the mean field.
