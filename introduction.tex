\chapter{Introduction}

%When you think of solids, you generally think of crystals. You take a liquid, and let it cool down, and it falls into a nice crystal structure that is spatially optimized in some way. This is generally how most solids form and is conceptually pretty easy to understand. Glass however can be formed by cooling down a liquid too quickly to find its minimum energy configuration - it essentially just gets stuck in some non-optimal state instead. 

Anyone who has stacked oranges knows that it is easy and intuitive to stack them closely in a dense lattice, as demonstrated in figure \ref{plot:oranges}(a). Filling about $74\%$ of space, the close packed lattice is in fact the densest possible arrangement of equal sized spheres in 3 dimensions. This was originally conjectured by Kepler in 1611 \cite{kepler_strena_1611}, but was not formally proven until 1998 \cite{hales_overview_1998}. If one simply pours them together, they will instead arrange themselves in an amorphous (i.e. random) and less dense structure, as shown in figure \ref{plot:oranges}(b). This amorphous configuration is significantly less efficient, filling only $0.62\%$ of space. The lazy but innovative produce worker may attempt to create a close lattice by randomly pouring oranges into a box and squeezing, but experience dictates that this is impossible. The fruit are too much in each other's way to rearrange and relieve the stress from the squeezing. Interestingly, 2 dimensional discs of equal size do spontanously form hexagonal lattices when squeezed, but discs of nonuniform size generally become frustrated and arrange amorphously.

\begin{figure}[b!]

\includegraphics[width=\columnwidth, trim=0 0 0 0, clip]{oranges.pdf}

\caption{Left, stack of oranges constructed in a close packed lattice. Right, oranges arranged amorphously. \jamesemph{[how to cite these appropriately?]}}
%https://unsplash.com/photos/A4BBdJQu2co
%https://www.tate.org.uk/art/artworks/louw-soul-city-pyramid-of-oranges-t13881

\label{plot:oranges}
\end{figure}

Granular materials are collections of distinct macroscopic objects, such as sand, ball bearings, or piles of oranges. While crystalline structures such close lattices are relatively straightforward to understand and analyze, the random features of amorphous granular systems make them extremely difficult to understand from first principles. The only analytic approach that has been somewhat successful is by taking the limit of infinite spatial dimensions, termed ``mean field.'' Since all physical systems are finite dimensional, results from the mean field limit do not necessarily apply. Many results from the mean field are however surprisingly accurate even in as low as 2 and 3 dimensions. Our research is not focused on analytic theory, but rather on computational study of soft spheres, i.e. spheres which are able to interact with their neighbors, imparting variable forces upon them. %Although the forces between granules of physical systems arise from physical deformations, this can be approximated with non-deforming but slightly overlapping spheres. 
In general when we simulate systems in greater than three dimensions, the purpose is to draw parallels with results from the mean field.


A phase transition is when a material undergoes a discontinuous change in some property (the ``order parameter'') as some other property (the ``control parameter'') varies. The most well known phase changes are freezing/melting and boiling/condensating, for which the temperature is the control parameter, and the density of the material can be seen to discontinuously change at critical temperatures. The order parameter is generally accepted to be the free energy - at the phase transition, melting absorbs a latent heat separate from the heat required just to heat it to the melting point. While glasses go through thermal phase transitions, granular materials do not. Granular materials are thus described as being ``athermal'' or ``zero-temperature.'' If you imagine the box of oranges as described earlier, thermal fluctuations (providing energy $E \sim kT \sim 0.026$ eV) will never cause rearrangements of a pair of oranges (requiring $E \sim mgr \sim 10^{17}$ eV). Granular materials do however go through a different type of transition, called the jamming and unjamming transitions, which are somewhat analogous to thermodynamic phase transitions. The control parameter for the jamming transition is the density $\varphi$, and the jamming transition happens at a critical density $\varphi_j$, where properties of the system go through discontinuous changes. In particular, the pressure $P$ goes from 0 to a finite value, and the number of force bearing contacts between granules goes from 0 to roughly the number of particles times the spatial dimension $Nd$. The critical density is an exact value for any particular system, but generally has some variation between systems, and can vary based on the protocol of system generation. The jamming point can be understood as a critical point, and as the system increases in density from that point, properties such as the number of contacts in excess of Nd, the pressure, and the density in excess of $\varphi_j$ all scale with each other as power laws \cite{ohern_jamming_2003,goodrich_scaling_2016}.

%Traditional approaches to analyzing thermodynamic systems rely on the assumption of ergodicity, i.e. that the thermal system explores the ensemble of all available configurations. The probability of existing in any particualar configuration is related to the energy of that configuration compared with the temperature of the system. In a zero temperature granular material, however, this is obviously impossible. Despite this, granular materials respond predictably to repeated disturbances, e.g. if you repeatedly shake a box of rocks, they will reliably settle into a configuration of roughly the same density. Thus some version of athermal statistical mechanics is clearly at work. In 1989 Sir Sam Edwards begun the first serious attempt at understanding granular materials in thermodynamic terms \cite{edwards_theory_1989}. In particular, a quantity termred ``angoricity,'' relates entropy to pressure rather than energy, termed ``angoricity,'' has shown promise as a temperature analog for granular materials \cite{edwards_distribution_2008,bi_statistical_2015}.

The jamming transition identifies the onset of rigidity. Below $\varphi_j$, a system is floppy and unable to sustain any force. Above $\varphi_j$ however a system becomes rigid and can support external forces. You can run your hands through sand at the beach, but if you compress it, it will push back. This rigidity can be understood in terms of degrees of freedom and constaints - it arises when the system has at least as many constraints as degrees of freedom. A system where these are exactly equal is termed ``isostatic,'' and a system with more constraints than degrees of freedom is termed ``hyperstatic.'' Since each particle can move in $d$ dimensions, a granular system has $Nd$ degrees of freedom. The constraints on those degrees of freedom are the contacts between particles, which increase in quantity with increasing density or pressure. The jammed configurations that we study are hyperstatic and thus are rigid and may support external forces. When considering instead the force network of the system, each contact is a degree of freedom, and requiring force balance on each particle imposes $Nd$ constraints. Thus while the positions of particles in a hyperstatic system is overdetermined, the force network supporting this rigidity is underdetermined. There exists a degerate space of allowed force configurations, which is referred to as the Force Network Ensemble or FNE \cite{snoeijer_force_2004,tighe_force_2010}. This Force Network Ensemble framework forms the basis for the majority of the material in this work.



The following three chapters are the three papers which we composed during my time as a graduate student. The first explores a method that we developed that uses the Force Network Ensemble to measure the allowed space of force configurations. This allowed us to measure the entropy of the force networks of a granular system. In the second, we use a similar approach to examine the boundaries of the allowed space of force configurations. These boundaries correspond to systems with fewer contacts than the original system. By examining which contacts are missing in those systems, we are able to predict which contacts between particles are unnecessary. We then show that these contacts are in fact broken under decompression. In the third, we closely examine the relation between a few critical power laws from the jamming transition. Rather than looking at the well known exponents of these power laws, we explore the prefactors to these power laws. Although pefactors such as these are generally highly sensitive to finite dimensional corrections, we show that they are still well predicted by the mean field.
